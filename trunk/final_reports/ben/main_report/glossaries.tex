\newglossaryentry{Compression}
 {name={Compression},
  description={Removal of redundancy according to some statistical model of the data. This redundancy may be repeated values or wasteful character encoding practices}
 }
\newglossaryentry{Decompression}
 {name={Decompression},
  description={The inverse of compression. Unpacks a file to its original state or a similar representation thereof (depending whether lossy or lossless compression is
  desired}
 }
\newglossaryentry{KAT-7}
 {name={KAT-7},
  description={Initial 7-dish prototype radio telescope array constructed near Carnarvon, South Africa}
 }
\newglossaryentry{MeerKAT}
 {name={MeerKAT},
  description={Successor to the KAT-7 prototype}
 }
\newglossaryentry{Square Kilometer Array (SKA)}
 {
  name={Square Kilometer Array},
  description={The combined low and high frequency radio telescope arrays constructed in Southern Africa and Australia will be collectively known as the SKA}
 }
\newglossaryentry{Petabyte} 
 {
  name={Petabyte},
  description={1024 Terrabytes (TiB)}
 }
\newglossaryentry{Pixel} 
 {
  name={Pixel},
  description={Picture element (or colour value). An image is a discretized raster of these pixels}
 }
\newglossaryentry{Scanline}
 {
  name={Scanline},
  description={A horizontal row of pixels that stretches across the width of the image}
 }
\newglossaryentry{Run-Length Encoding (RLE)}
 {
  name={Run-Length Encoding (RLE)},
  description={A technique that clusters runs of symbols into length-based representations (aaabbbb is represented as 3a4b for example). See the introduction}
 }
\newglossaryentry{Lossless Compression}
 {
  name={Lossless Compression},
  description={Can be inverted with no loss of information. The opposite will be lossy compression}
 }
\newglossaryentry{Online Compression}
 {
  name={Online Compression},
  description={Compression is completed as part of the primary function of the system (and not afterwards). An online process is normally required to
  add no significant latencies to an existing system}
 }
\newglossaryentry{Quantization}
 {
  name={Quantization},
  description={In terms of lossy compression this process refers to binning values in close proximity into groups}
 }
\newglossaryentry{Compression ratio}
 {
  name={Compression ratio},
  description={Ratio between output size and input size}
 }
\newglossaryentry{Throughput}
 {
  name={Throughput},
  description={input processed in GiB per second}
 }
\newglossaryentry{Graphical Processing Unit (GPU)}
 {
  name={Graphical Processing Unit (GPU)},
  description={Independent streaming processor that is tailored towards processing graphics}
 }
\newglossaryentry{General Purpose Graphical Processing Unit (GPU)}
 {
  name={General Purpose Graphical Processing Unit Programming (GPGPU)},
  description={General purpose (for instance scientific computing) programming using SIMD style instructions on a GPU}
 }
\newglossaryentry{Simple Instruction Multiple Data (SIMD)}
 {
  name={Simple Instruction Multiple Data (SIMD)},
  description={Can apply to contexts where a large number of elements are subject to the same instruction. A good example where this style of programming
  can be used is vector addition}
 }
\newglossaryentry{Lempel-Ziv (LZ) Compression}
 {
  name={Lempel-Ziv (LZ) Compression},
  description={A compression scheme where repeated sequences of symbols are encoded as distance-length pairs. See background chapter}
 }
\newglossaryentry{Entropy}
 {
  name={Entropy},
  description={The measurement of the information contained in a single base-n symbol (transmitted per unit time by some source)}
 }
\newglossaryentry{Redundancy}
 {
  name={Redundancy},
  description={The difference between the optimal entropy of a dataset and the actual entropy}
 }
\newglossaryentry{Huffman coding}
 {
  name={Huffman coding},
  description={An entropy encoder that assigns shorter codes to frequently used symbols in a dataset. See background chapter}
 }
\newglossaryentry{Arithmetic coding}
 {
  name={Arithmetic coding},
  description={An entropy encoder that assigns sub-intervals of [0,1) to the symbols it processes. This encoder is closer to an optimal entropy encoder, achieving better compression than 
  Huffman coding. See background chapter}
 }
\newglossaryentry{Discrete Cosine Transform (DCT)}
 {
  name={Discrete Cosine Transform (DCT)},
  description={A finite sequence of samples can be represented as a sum of transformed cosine functions. This is one example of where transforms can be used to
  achieve level-of-detail-based compression. See background chapter}
 }
\newglossaryentry{Compute Unified Device Architecture (CUDA)}
 {
  name={Compute Unified Device Architecture (CUDA)},
  description={A programming framework used to compile programs for Nvidia GPUs}
 } 
\newglossaryentry{Lagrange predictor}
 {
  name={Lagrange predictor},
  description={A polynomial extrapolation technique used for prediction of successive values for smooth functions. See design section}
 } 
\newglossaryentry{Lorenzo predictor}
 {
  name={Lorenzo predictor},
  description={A scheme that extends the parallelogram rule to higher dimensions which can be used for polynomial reconstruction. See design section}
 } 
\newglossaryentry{IEEE 754}
 {
  name={IEEE 754},
  description={A widely employed standard defined for the storage of 32-bit, 64-bit or 128-bit floating point values. See background chapter}
 } 
\newglossaryentry{Symmetrical algorithm}
 {
  name={Symmetrical algorithm},
  description={In terms of compression and decompression this refers to executing a compression algorithm in reverse in order to achieve decompression}
 }
\newglossaryentry{Prefix sum}
 {
  name={Prefix sum},
  description={An element-wise accumulation of an array of values over some binary associative operator. See design chapter}
 }
\newglossaryentry{Race condition}
 {
  name={Race condition},
  description={A condition that occurs when two threads compete to update the same resource in which correctness depends on the order and timing of these updates. This 
  constitutes non-deterministic behavior which invalidates the state of such a resource. See design chapter on how this affects a parallel packing algorithm}
 }
\newglossaryentry{Streaming Multiprocessor (SM)}
 {
  name={Streaming Multiprocessor (SM)},
  description={A sub-component of Nvidia GPU architecture. Such a multiprocessor consists of many arithmetic units which operate in parallel. A GPU will normally have
  multiple of these multiprocessors which will operate independently of eachother unless a global synchronization barrier is imposed. See design chapter}
 }
\newglossaryentry{Warp of threads}
 {
  name={Warp of threads},
  description={In a GPU each Streaming Multiprocessor is divided into atomic groups of threads that executes in parallel. See design chapter}
 }
\newglossaryentry{Shared memory}
 {
  name={Shared memory},
  description={In CPUs shared memory refers to the sharing of primary memory between multiple processor cores, or even multiple processors. In GPUs the threads within
  a single SM has access to dedicated fast memory that can be used to avoid calls to global device memory. See design chapter}
 }
\newglossaryentry{Coalesced memory access}
 {
  name={Coalesced memory access},
  description={On a GPU a coalesced read occurs when the global memory calls of multiple threads are grouped together in order to share the latency of such a call
  between threads. This normally requires memory aligned access patterns. See design chapter}
 }
\newglossaryentry{Bank conflict}
 {
  name={Bank conflict},
  description={On a GPU a bank conflict occurs when multiple threads try to access the same bank of shared memory. Such accesses will then be completed sequentially.
  See design chapter}
 }
\newglossaryentry{Open Systems Interconnection (OSI) network stack}
 {
  name={Open Systems Interconnection (OSI) network stack},
  description={An abstraction of the layered nature of network protocols. From top to bottom the stack contains the following layers: application, presentation, session,
  transport, network, link and physical layers. A decompression operation may form part of the presentation layer in such a stack}
 }
\newglossaryentry{Streaming SIMD Extensions (SSE)}
 {
  name={Streaming SIMD Extensions (SSE)},
  description={A vectorized instruction set (see design and implementation chapters) that adds the capability of performing SIMD instructions on 128-bit registers (for example
  4 32-bit integers) in one machine clock cycle}
 }
\newglossaryentry{eXtended Operations (XOP)}
 {
  name={eXtended Operations (XOP)},
  description={An AMD extension to SSE instruction set that adds, amongst others, the ability to perform per-element bit-shifting on 4 integers contained in a 128-bit 
  register}
 }
\newglossaryentry{Advanced Vector Extensions (AVX)}
 {
  name={Advanced Vector Extensions (AVX)},
  description={An Intel extension to SSE instruction set that adds 256-bit registers (which may be used to perform simultaneous operations on up to 8 32-bit 
  integers/floating-point values)}
 }
\newglossaryentry{GNU ZIP (GZIP)}
 {
  name={GNU ZIP (GZIP)},
  description={A standard free compression utility that implements the DEFLATE (LZ-77 based) algorithm}
 }
\newglossaryentry{BZIP2}
 {
  name={BZIP2},
  description={A standard free compression utility that implements the Burrows-Wheeler Transform (which groups matching elements into runs, in order to perform
  Run-Length Encoding), along with Huffman encoding}
 }
\newglossaryentry{ZIP}
 {
  name={ZIP},
  description={A standard compression utility that implements a collection of LZ, entropy and transform methods that is useful when archiving a group of files}
 }
\newglossaryentry{Decoupling}
 {
  name={Decoupling},
  description={A software engineering term used to measure the dependency of one component on another}
 }
\newglossaryentry{Cohesion}
 {
  name={Cohesion},
  description={A software engineering term used to measure the the degree to which a component separates operations. Ideally a component should only house supporting
  operations to perform a single task to ease future maintenance and usability}
 }
\newglossaryentry{Exclusive Or (XOR)}
 {
  name={Exclusive Or (XOR)},
  description={A bitwise, binary (taking two inputs), operation that returns false if the inputs are identical}
 }
\newglossaryentry{Pivot}
 {
  name={Pivot},
  description={A pivot divides a dataset into 2 groups (useful in applications of fast sorting techniques). One group is larger than the pivot, while the other is smaller (or equal)
  to the pivot}
 }
\newglossaryentry{OpenMP}
 {
  name={OpenMP},
  description={A package that allows for easy parallelization of C and Fortran code using compiler flags}
 }
\newglossaryentry{POSIX threads (pthreads)}
 {
  name={POSIX threads (pthreads)},
  description={An API that allows for the creation of native process-based threading on Unix-like systems}
 }
\newglossaryentry{Intel Multimedia eXtensions (MMX)}
 {
  name={Intel Multimedia eXtensions (MMX)},
  description={A vectorized instruction set that added support for performing operations on two 32-bit integers through the 64-bit registers this technology added.
  The MMX instruction set was extended by the AMD 3DNow! extensions and later the SSE extensions}
 }
\newglossaryentry{3DNow!}
 {
  name={3DNow!},
  description={An AMD vectorized instruction set that extended the capabilities of Intel's MMX instruction set by adding supporting instructions for 32-bit floating-point 
  processing}
 }
\newglossaryentry{Pinned host memory}
 {
  name={Pinned host memory},
  description={A method used to allocate a non-swappable block of primary memory that improves memory transfers between a host machine and a GPU}
 }
\newglossaryentry{Zero-copy}
 {
  name={Zero-copy},
  description={A support mechanism in the Nvidia API that allows for sharing pinned host, and device memory spaces}
 } 
\newglossaryentry{User Datagram Protocol (UDP)}
 {
  name={User Datagram Protocol (UDP)},
  description={A fast networking protocol that sends data packets with no built-in support for verifying package integrity, and checking (and by extension correcting) 
  transmission losses}
 }
\newglossaryentry{Merkle Damgard 5 (MD5)}
 {
  name={Merkle Damgard 5 (MD5)},
  description={An iterative message digest algorithm that creates a fixed-length checksum of a message, with the intension of verifying data integrity. See section
  on approach feasibility}
 }
\newglossaryentry{HDF5 file format}
 {
  name={HDF5 file format},
  description={A hierarchical file format that supports the storage of large sets of numerical data and associated meta-data that is generally faster than
  storage using relational databases\footnote{The HDF Group. Hierarchical data format version 5, 2000-2010. http://www.hdfgroup.org/HDF5.}}
 }